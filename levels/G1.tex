%%p01.10s
Un tri\'angulo tiene $\underline{\phantom{3}}$ v\'ertices.
%%a01
\begin{enum}
	* $4$
	* $6$
	* $3$
	* $8$
\end{enum}
%%r01
$3$
%%p02.10s
Tri\'angulo o tambi\'en llamado...?
%%a02
\begin{enum}
	* Ala\'in
	* Tr\'igono
	* Triangulitis
	* Cuadril\'atero
\end{enum}
%%r02
Tr\'igono
%%p03.10s
En la figura, hallar ``$x$''.
\begin{figure}[h]
	\begin{tikzpicture}[thick]
		\def\r{2.5}
		\tkzDefPoint(-20:\r){A}
		\tkzDefPoint(180:\r){B}
		\tkzDefPoint(-98:\r){C}
		\tkzDefBarycentricPoint(A=-2,C=5) \tkzGetPoint{D}
		\def\a{6mm}
		\tkzFillAngles[size=\a,fill=blue,opacity=.2](B,A,C C,B,A B,C,D)
		\tkzMarkAngles[size=\a,mark=none](B,A,C C,B,A B,C,D)
		\tkzLabelAngle(B,A,C){$41\dg$}
		\tkzLabelAngle(C,B,A){$x$}
		\tkzLabelAngle(B,C,D){$80\dg$}
		\tkzDrawPolygon(A,B,C)
		\tkzDrawSegment[dashed](C,D)
		\tkzLabelPoints[right](A)
		\tkzLabelPoints[left](B)
		\tkzLabelPoints[below](C)
	\end{tikzpicture}
\end{figure}
%%a03
\begin{task}
	* $52\dg$
	* $39\dg$
	* $61\dg$
	* $121\dg$
\end{task}
%%r03
$39\dg$
%%p04.40s
Calcula el valor de ``$\alpha$''.
	\begin{figure}[h]
		\begin{tikzpicture}[thick]
			\def\r{3}
			\tkzDefPoint(-165:\r){A}
			\tkzDefPoint(59:\r){B}
			\tkzDefPoint(-15:\r){C}
			\tkzDefPointOnLine[pos=1.2](A,C)
			\tkzGetPoint{D}
			\tkzFillAngles[size=5mm,fill=blue,opacity=.2](C,A,B A,B,C D,C,B)
			\tkzMarkAngles[size=5mm,mark=none](C,A,B A,B,C D,C,B)
			\tkzLabelAngles[pos=1](C,A,B){$37\dg$}
			\tkzLabelAngle[pos=.9](A,B,C){$\alpha$}
			\tkzLabelAngle[pos=1](D,C,B){$112\dg$}
			\tkzDrawPolySeg(D,A,B,C)
		\end{tikzpicture}
	\end{figure}
%%a04
\begin{task}
	* $78\dg$
	* $75\dg$
	* $73\dg$
	* $74\dg$
\end{task}
%%r04
$75\dg$
%%p05.sp.40s
\begin{mini}
	El per\'imetro del tri\'angulo es de $\ce{20cm}$. ¿Cu\'anto mide el lado $AC$?
	\begin{center}
		\begin{tikzpicture}[thick]
			\def\r{3}
			\def\a{16.6}
			\tkzDefPoint(\a+146.8:\r){A}
			\tkzDefPoint(\a:\r){B}
			\tkzDefPoint(\a+50.42:\r){C}
			\draw (A) node[below left] {$A$} -- node[below] {$x+5$} (B) node[below right] {$B$} -- node[above right] {$x$} (C) node[above] {$C$} -- node[above left] {$x+3$} (A);
		\end{tikzpicture}
	\end{center}
\end{mini}
%%a05
\begin{task}
	* $\ce{5cm}$
	* $\ce{2cm}$
	* $\ce{7cm}$
	* $\ce{6cm}$
\end{task}
%%r05
$\ce{7cm}$
%%p06.40s
En la figura, hallar ``$x$''.
\begin{figure}[h]
	\begin{tikzpicture}[thick]
		\def\r{2.5}
		\tkzDefPoint(-150:\r){A}
		\tkzDefPoint(-30:\r){B}
		\tkzDefPoint(60:\r){C}
		\def\a{6mm}
		\tkzFillAngles[size=\a,fill=blue,opacity=.2](B,A,C C,B,A A,C,B)
		\tkzMarkAngles[size=\a,mark=none](B,A,C C,B,A A,C,B)
		\tkzLabelAngle(B,A,C){$3x$}
		\tkzLabelAngle(C,B,A){$5x$}
		\tkzLabelAngle(A,C,B){$4x$}
		\tkzDrawPolygon(A,B,C)
		\tkzLabelPoints[below left](A)
		\tkzLabelPoints[below right](B)
		\tkzLabelPoints[above](C)
	\end{tikzpicture}
\end{figure}
%%a06
\begin{task}
	* $15\dg$
	* $18\dg$
	* $20\dg$
	* $10\dg$
\end{task}
%%r06
$15\dg$
%%p07.100s
Si $ABCD$ es un paralelogramo, calcular ``$x$''.
\begin{figure}[h]
	\begin{tikzpicture}[thick]
		\def\x{4}
		\def\y{3.8}
		\tkzDefPoints{0/0/A,\x/0/D}
		\tkzDefPoint(144:\y){B}
		\tkzDefParallelogram(D,A,B) \tkzGetPoint{C}
		\tkzFillAngles[size=5mm,fill=blue,opacity=.2](B,C,D C,D,A)
		\tkzMarkAngles[size=5mm,mark=none](B,C,D C,D,A)
		\tkzLabelAngle[pos=.8](B,C,D){$4x$}
		\tkzLabelAngle(C,D,A){$x$}
		\tkzDrawPolygon(A,B,C,D)
		\tkzLabelPoints[below left](A)
		\tkzLabelPoints[above left](B)
		\tkzLabelPoints[above right](C)
		\tkzLabelPoints[below right](D)
	\end{tikzpicture}
\end{figure}
%%a07
\begin{task}
	* $24\dg$
	* $36\dg$
	* $20\dg$
	* $72\dg$
\end{task}
%%r07
$36\dg$
%%p08.40s
En el romboide $ABCD$, calcular ``$x$''.
\begin{figure}[h]
	\begin{tikzpicture}[thick]
		\tkzDefPoints{0/0/A,4/0/D}
		\tkzDefPoint(75:3){B}
		\tkzDefParallelogram(B,A,D) \tkzGetPoint{C}
		\def\r{6mm}
		\tkzFillAngles[size=\r,fill=blue,opacity=.2](A,B,C B,C,D)
		\tkzMarkAngles[size=\r,mark=none](A,B,C B,C,D)
		\tkzLabelAngle(A,B,C){$7x$}
		\tkzLabelAngle(B,C,D){$5x$}
		\tkzDrawPolygon(A,B,C,D)
		\tkzLabelPoints[below left](A)
		\tkzLabelPoints[above left](B)
		\tkzLabelPoints[above right](C)
		\tkzLabelPoints[below right](D)
	\end{tikzpicture}
\end{figure}
%%a08
\begin{task}
	* $12\dg$
	* $15\dg$
	* $10\dg$
	* $8\dg$
\end{task}
%%r08
$15\dg$
%%p09.sp.30s
\begin{mini}
	¿Cu\'antas diagonales se pueden trazar en un hex\'agono?
\end{mini}
%%a09
\begin{enum}
	* $6$
	* $9$
	* $12$
	* $7$
\end{enum}
%%r09
$9$
%%p10.sp.70s
\begin{mini}
	¿Cu\'antas diagonales se pueden trazar en total en un pol\'igono de $28$ lados?
\end{mini}
%%a10
\begin{enum}
	* $350$
	* $250$
	* $170$
	* $280$
\end{enum}
%%r10
$350$
%%p11.sp.50s
\begin{mini}
	Calcula la suma de las medidas de los \'angulos internos de un dec\'agono.
\end{mini}
%%a11
\begin{enum}
	* $1080\dg$
	* $1260\dg$
	* $1440\dg$
	* $900\dg$
\end{task}
%%r11
$1440\dg$
%%p12.sp.40s
\begin{mini}
	¿Cu\'anto mide el \'angulo externo de un icos\'agono regular?
\end{mini}
%%a12
\begin{enum}
	* $12\dg$
	* $16\dg$
	* $10\dg$
	* $18\dg$
\end{enum}
%%r12
$18\dg$
%%p13.sp.70s
\begin{mini}
	¿Cu\'anto suman los \'angulos externos e internos de un icos\'agono?
\end{mini}
%%a13
\begin{enum}
	* $2800\dg$
	* $8200\dg$
	* $3600\dg$
	* $4000\dg$
\end{enum}
%%r13
$3600\dg$
%%p14.sp.90s
\begin{mini}
	Si se triplica el n\'umero de lados de un pol\'igono, la suma de sus \'angulos internos queda quintuplicada. ¿C\'omo se llama dicho pol\'igono?
\end{mini}
%%a14
\begin{enum}
	* Tri\'angulo
	* Ala\'in
	* Cuadril\'atero
	* Oct\'ogono
\end{enum}
%%r14
Cuadril\'atero
%%p15.sp.70s
\begin{mini}
	Hallar la suma de los n\'umeros de diagonales de un non\'agono y de un endec\'agono.
\end{mini}
%%a15
\begin{task}
	* $27$
	* $44$
	* $71$
	* $69$
\end{task}
%%r15
$71$
