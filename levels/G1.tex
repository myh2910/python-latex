%%p01.sp
\begin{mini}
	Calcula la suma de las medidas de los \'angulos internos de un dec\'agono.
\end{mini}
%%a01
\begin{enum}
	* $1080\dg$
	* $1260\dg$
	* $1440\dg$
	* $900\dg$
\end{task}
%%r01
$1440\dg$
%%p02.sp
\begin{mini}
	¿Cu\'antas diagonales se pueden trazar en un hex\'agono?
\end{mini}
%%a02
\begin{enum}
	* $6$
	* $9$
	* $12$
	* $7$
\end{enum}
%%r02
$9$
%%p03.sp
\begin{mini}
	¿Cu\'anto mide el \'angulo externo de un icos\'agono regular?
\end{mini}
%%a03
\begin{enum}
	* $12\dg$
	* $16\dg$
	* $10\dg$
	* $18\dg$
\end{enum}
%%r03
$18\dg$
%%p04.sp
\begin{mini}
	¿Cu\'anto suman los \'angulos externos e internos de un icos\'agono?
\end{mini}
%%a04
\begin{enum}
	* $2800\dg$
	* $8200\dg$
	* $3600\dg$
	* $4000\dg$
\end{enum}
%%r04
$3600\dg$
%%p05
Tri\'angulo o tambi\'en llamado...?
%%a05
\begin{enum}
	* Ala\'in
	* Tr\'igono
	* Triangulitis
	* Cuadril\'atero
\end{enum}
%%r05
Tr\'igono
%%p06
Un tri\'angulo tiene $\underline{\phantom{xxxxxxxx}}$ v\'ertices.
%%a06
\begin{enum}
	* $4$
	* $6$
	* $3$
	* $8$
\end{enum}
%%r06
$3$
%%p07
En la figura, hallar ``$x$''.
\begin{figure}[h]
	\begin{tikzpicture}
		\coordinate (A) at (-20:2.5);
		\coordinate (B) at (180:2.5);
		\coordinate (C) at (-98:2.5);
		\coordinate (D) at (barycentric cs:A=-2,C=5);
		{ [every path/.style={draw,line width=1pt,angle radius=6mm,angle eccentricity=1.7}]
			\pic["$41\dg$"] {angle=B--A--C};
			\pic["$x$"] {angle=C--B--A};
			\pic["$80\dg$"] {angle=B--C--D};
		}
		\draw[dashed,thick] (C) -- (D);
		\draw[thick] (A) node[right] {$A$} -- (B) node[left] {$B$} -- (C) node[below] {$C$} -- (A);
	\end{tikzpicture}
\end{figure}
%%a07
\begin{task}
	* $52\dg$
	* $39\dg$
	* $61\dg$
	* $121\dg$
\end{task}
%%r07
$39\dg$
%%p08
En la figura, calcular $\alpha$.
\begin{figure}[h]
	\begin{tikzpicture}
		\coordinate (A) at (-160:2.5);
		\coordinate (B) at (120:2.5);
		\coordinate (C) at (-30:2.5);
		{ [every path/.style={draw,line width=1pt,angle radius=6mm,angle eccentricity=1.7}]
			\pic["$75\dg$"] {angle=C--A--B};
			\pic["$\alpha$"] {angle=A--B--C};
			\pic["$40\dg$"] {angle=B--C--A};
		}
		\draw[thick] (A) node[below left] {$A$} -- (B) node[above] {$B$} -- (C) node[below right] {$C$} -- (A);
	\end{tikzpicture}
\end{figure}
%%a08
\begin{task}
	* $80\dg$
	* $40\dg$
	* $65\dg$
	* $150\dg$
\end{task}
%%r08
$65\dg$
%%p09
En la figura, hallar ``$x$''.
\begin{figure}[h]
	\begin{tikzpicture}
		\coordinate (A) at (-150:2.5);
		\coordinate (B) at (-30:2.5);
		\coordinate (C) at (60:2.5);
		{ [every path/.style={draw,line width=1pt,angle radius=6mm,angle eccentricity=1.7}]
			\pic["$3x$"] {angle=B--A--C};
			\pic["$5x$"] {angle=C--B--A};
			\pic["$4x$"] {angle=A--C--B};
		}
		\draw[thick] (A) node[below left] {$A$} -- (B) node[below right] {$B$} -- (C) node[above] {$C$} -- (A);
	\end{tikzpicture}
\end{figure}
%%a09
\begin{task}
	* $15\dg$
	* $18\dg$
	* $20\dg$
	* $10\dg$
\end{task}
%%r09
$15\dg$
%%p10.sp
\begin{mini}
	Escribe verdadero (V) o falso (F) seg\'un corresponda y encierra la alternativa correcta.
	\begin{center}\vspace{-12pt}
		\begin{tabular}{m{.01\linewidth}m{.77\linewidth}l}
			I.&Un hex\'agono posee $9$ diagonales.&(\phantom{V}) \vspace{4pt}\\
			II.&Un pent\'agono posee $5$ diagonales.&(\phantom{V}) \vspace{4pt}\\
			III.&La suma de las medidas de \'angulos&\multirow{2}{*}{(\phantom{V})} \\
			&interiores de un dodec\'agono es $1800\dg$.&
		\end{tabular}
	\end{center}
\end{mini}
%%a10
\begin{task}
	* VVF
	* FFV
	* VVV
	* FFF
\end{task}
%%r10
VVV
%%p11
\begin{mini}
	Hallar la medida del \'angulo interior del pol\'igono regular mostrado.
	\begin{center}
		\begin{tikzpicture}
			\node [draw,thick,regular polygon,regular polygon sides=5,minimum size=4cm] {};
		\end{tikzpicture}
	\end{center}
\end{mini}
%%a11
\begin{task}
	* $92\dg$
	* $120\dg$
	* $108\dg$
	* $140\dg$
\end{task}
%%r11
$108\dg$
%%p12
Calcular ``$x$''.
\begin{figure}[h]
	\begin{tikzpicture}
		\coordinate (v0) at (0,0);
		\coordinate (v1) at (0,4);
		\coordinate (v2) at (-6.4,0);
		\path [name path=p1] (v1) -- ++ (-171:6.4);
		\path [name path=p2] (v2) -- ++ (54:4.2);
		{ [every path/.style={draw,line width=1pt,angle radius=5mm,angle eccentricity=1.75},name intersections={of=p1 and p2, by={v3}}]
			\pic["$3x$"] {angle=v3--v1--v0};
			\pic["$2x$"] {angle=v0--v2--v3};
			\pic["$x$"] {angle=v2--v3--v1};
			\pic {right angle=v1--v0--v2};
		}
		\draw[thick] (v1) -- (v0) -- (v2) -- (v3) -- (v1);
	\end{tikzpicture}
\end{figure}
%%a12
\begin{task}
	* $13\dg$
	* $17\dg$
	* $14\dg$
	* $27\dg$
\end{task}
%%r12
$27\dg$
%%p13
Calcular ``$\theta$'' en:
\begin{figure}[h]
	\begin{tikzpicture}[thick]
		\def\r{4.5}
		\def\a{5mm}
		\tkzDefPoint(-160:\r){A}
		\tkzDefPoint(164:\r){B}
		\tkzDefPoint(-12:\r){C}
		\tkzFillAngles[size=\a,fill=blue,opacity=.2](C,A,B A,B,C)
		\tkzFillAngle[size=8mm,fill=blue,opacity=.2](B,C,A)
		\tkzMarkAngles[size=\a,mark=none](C,A,B A,B,C)
		\tkzMarkAngle[size=8mm,mark=none](B,C,A)
		\tkzLabelAngle[shift={(0,-.1)}](C,A,B){$2\theta$}
		\tkzLabelAngle[shift={(.2,0)}](A,B,C){$\theta+30\dg$}
		\tkzLabelAngle[pos=1.7](B,C,A){$18\dg$}
		\tkzDrawPolygon(A,B,C)
	\end{tikzpicture}
\end{figure}
%%a13
\begin{task}
	* $42\dg$
	* $45\dg$
	* $44\dg$
	* $48\dg$
\end{task}
%%r13
$44\dg$
%%p14
En el romboide $ABCD$, calcular ``$x$''.
\begin{figure}[h]
	\begin{tikzpicture}[thick]
		\tkzDefPoints{0/0/A,4/0/D}
		\tkzDefPoint(75:3){B}
		\tkzDefParallelogram(B,A,D) \tkzGetPoint{C}
		\def\r{6mm}
		\tkzFillAngles[size=\r,fill=blue,opacity=.2](A,B,C B,C,D)
		\tkzMarkAngles[size=\r,mark=none](A,B,C B,C,D)
		\tkzLabelAngle(A,B,C){$7x$}
		\tkzLabelAngle(B,C,D){$5x$}
		\tkzDrawPolygon(A,B,C,D)
		\tkzLabelPoints[below left](A)
		\tkzLabelPoints[above left](B)
		\tkzLabelPoints[above right](C)
		\tkzLabelPoints[below right](D)
	\end{tikzpicture}
\end{figure}
%%a14
\begin{task}
	* $12\dg$
	* $15\dg$
	* $10\dg$
	* $8\dg$
\end{task}
%%r14
$15\dg$
%%p15.sp
\begin{mini}
	¿Cu\'antas diagonales se pueden trazar en total en un pol\'igono de $28$ lados?
\end{mini}
%%r15
$350$
