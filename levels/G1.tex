%%p01.sp
\begin{mini}
	Calcula la suma de las medidas de los \'angulos internos de un dec\'agono.
\end{mini}
%%a01
\begin{task}
	*
	*
	*
	*
\end{task}
%%p02.sp
\begin{mini}
	¿Cu\'antas diagonales se pueden trazar en un hex\'agono?
\end{mini}
%%p03.sp
\begin{mini}
	¿Cu\'anto mide el \'angulo externo de un icos\'agono regular?
\end{mini}
%%p04.sp
\begin{mini}
	¿Cu\'anto suman los \'angulos externos e internos de un icos\'agono?
\end{mini}
%%p05
Tri\'angulo o tambi\'en llamado...?
%%p06
Un tri\'angulo tiene $\underline{\phantom{xxxxxxxx}}$ v\'ertices.
%%p07
En la figura, hallar ``$x$''.
\begin{figure}[h]
	\begin{tikzpicture}
		\coordinate (A) at (-20:2.5);
		\coordinate (B) at (180:2.5);
		\coordinate (C) at (-98:2.5);
		\coordinate (D) at (barycentric cs:A=-2,C=5);
		{ [every path/.style={draw,line width=1pt,angle radius=6mm,angle eccentricity=1.7}]
			\pic["$41\dg$"] {angle=B--A--C};
			\pic["$x$"] {angle=C--B--A};
			\pic["$80\dg$"] {angle=B--C--D};
		}
		\draw[dashed,thick] (C) -- (D);
		\draw[thick] (A) node[right] {$A$} -- (B) node[left] {$B$} -- (C) node[below] {$C$} -- (A);
	\end{tikzpicture}
\end{figure}
%%p08
En la figura, calcular $\alpha$.
\begin{figure}[h]
	\begin{tikzpicture}
		\coordinate (A) at (-160:2.5);
		\coordinate (B) at (120:2.5);
		\coordinate (C) at (-30:2.5);
		{ [every path/.style={draw,line width=1pt,angle radius=6mm,angle eccentricity=1.7}]
			\pic["$75\dg$"] {angle=C--A--B};
			\pic["$\alpha$"] {angle=A--B--C};
			\pic["$40\dg$"] {angle=B--C--A};
		}
		\draw[thick] (A) node[below left] {$A$} -- (B) node[above] {$B$} -- (C) node[below right] {$C$} -- (A);
	\end{tikzpicture}
\end{figure}
%%p09
En la figura, hallar ``$x$''.
\begin{figure}[h]
	\begin{tikzpicture}
		\coordinate (A) at (-150:2.5);
		\coordinate (B) at (-30:2.5);
		\coordinate (C) at (60:2.5);
		{ [every path/.style={draw,line width=1pt,angle radius=6mm,angle eccentricity=1.7}]
			\pic["$3x$"] {angle=B--A--C};
			\pic["$5x$"] {angle=C--B--A};
			\pic["$4x$"] {angle=A--C--B};
		}
		\draw[thick] (A) node[below left] {$A$} -- (B) node[below right] {$B$} -- (C) node[above] {$C$} -- (A);
	\end{tikzpicture}
\end{figure}
%%p10.sp
\begin{mini}
	Escribe verdadero (V) o falso (F) seg\'un corresponda y encierra la alternativa correcta.
	\begin{center}\vspace{-12pt}
		\begin{tabular}{m{.01\linewidth}m{.77\linewidth}l}
			I.&Un hex\'agono posee $9$ diagonales.&(\phantom{V}) \vspace{4pt}\\
			II.&Un pent\'agono posee $5$ diagonales.&(\phantom{V}) \vspace{4pt}\\
			III.&La suma de las medidas de \'angulos&\multirow{2}{*}{(\phantom{V})} \\
			&interiores de un dodec\'agono es $1800\dg$.&
		\end{tabular}
	\end{center}
\end{mini}
%%p11
\begin{mini}
	Hallar la medida del \'angulo interior del pol\'igono regular mostrado.
	\begin{center}
		\begin{tikzpicture}
			\node [draw,thick,regular polygon,regular polygon sides=5,minimum size=4cm] {};
		\end{tikzpicture}
	\end{center}
\end{mini}
