%%p01.sp
\begin{mini}
	Si $CD=\ce{10u}$, hallar la longitud del segmento que une los puntos medios de $\ol{AC}$ y $\ol{BD}$.
	\begin{center}
		\begin{tikzpicture}[thick]
			\def\r{2.5}
			\tkzDefPoints{0/0/A,0/\r/B,\r/\r/C,1/0/D1}
			\tkzDefPointBy[rotation=center C angle 143](B) \tkzGetPoint{D2}
			\tkzInterLL(A,D1)(C,D2) \tkzGetPoint{D}
			\tkzMarkRightAngles[size=.3,fill=green!20](D,A,B A,B,C)
			\tkzFillAngle[size=6mm,fill=green,opacity=.2](C,D,A)
			\tkzMarkAngle[size=6mm,mark=none](C,D,A)
			\tkzLabelAngle[pos=1.1](C,D,A){$37\dg$}
			\tkzDrawPolygon(A,B,C,D)
			\tkzLabelPoints[below left](A)
			\tkzLabelPoints[above left](B)
			\tkzLabelPoints[above right](C)
			\tkzLabelPoints[below right](D)
		\end{tikzpicture}
	\end{center}
\end{mini}
%%a01
\begin{task}
	* $\ce{4u}$
	* $\ce{3u}$
	* $\ce{3,5u}$
	* $\ce{6u}$
\end{task}
%%r01
$\ce{4u}$
%%p02
En la figura, $\ol{BC}\parallel\ol{AD}$. Calcular ``$\alpha$'' y ``$\beta$''.
\begin{figure}[h]
	\begin{tikzpicture}[thick]
		\def\x{3}
		\def\y{.8}
		\tkzDefPoints{\y/0/O1,-\y/0/O2}
		\tkzDefPoint(110:\x){B1}
		\tkzDefPoint(70:\x){C1}
		\tkzDefParallelogram(C1,O,O1) \tkzGetPoint{C}
		\tkzDefParallelogram(O2,O,B1) \tkzGetPoint{B}
		\tkzDefParallelogram(O1,C,B) \tkzGetPoint{A}
		\tkzDefParallelogram(C,B,O2) \tkzGetPoint{D}
		\tkzFillAngles[size=4mm,fill=green,opacity=.2](D,A,B A,B,C B,C,D C,D,A)
		\tkzMarkAngles[size=4mm,mark=none](D,A,B A,B,C B,C,D C,D,A)
		\tkzLabelAngle(D,A,B){$70\dg$}
		\tkzLabelAngle[pos=.8](A,B,C){$\beta$}
		\tkzLabelAngle[shift={(-.2,.1)}](B,C,D){$2\alpha-5\dg$}
		\tkzLabelAngle[shift={(-.2,-.1)}](C,D,A){$\alpha+5\dg$}
		\tkzDrawPolygon(A,B,C,D)
		\tkzLabelPoints[below left](A)
		\tkzLabelPoints[above left](B)
		\tkzLabelPoints[above right](C)
		\tkzLabelPoints[below right](D)
	\end{tikzpicture}
\end{figure}
%%a02
\begin{mini}[1.4]
	\begin{task}
		* $124\dg$ y $12\dg$
		* $36\dg$ y $20\dg$
		* $120\dg$ y $70\dg$
		* $110\dg$ y $60\dg$
	\end{task}
\end{mini}
%%r02
$110\dg$ y $60\dg$
%%p03
En la figura, calcular ``$x$''.
\begin{figure}[h]
	\begin{tikzpicture}[thick]
		\def\r{3.5}
		\tkzDefPoint(170:\r){A}
		\tkzDefPoint(100:\r){B}
		\tkzDefPoint(10:\r){C}
		\tkzInCenter(A,B,C) \tkzGetPoint{I}
		\tkzDefPointsBy[homothety=center C ratio 1.2](B,I){B1,I1}
		\tkzFillAngles[size=6mm,fill=green,opacity=.2](B,C,I I,C,A C,A,I I,A,B)
		\tkzMarkAngles[size=6mm,mark=|](B,C,I I,C,A)
		\tkzMarkAngles[size=5.4mm,arc=ll,mark=none](C,A,I I,A,B)
		\tkzFillAngles[size=4mm,fill=green,opacity=.2](B1,B,A I1,I,A)
		\tkzMarkAngles[size=4mm,mark=none](B1,B,A I1,I,A)
		\tkzDrawPolygon(A,B,C)
		\tkzDrawSegment(B1,B)
		\tkzDrawSegment(A,I)
		\tkzDrawSegment(I1,C)
		\tkzLabelAngles[pos=1.2](B,C,I I,C,A){$\theta$}
		\tkzLabelAngles(C,A,I I,A,B){$\alpha$}
		\tkzLabelAngle[pos=.7](B1,B,A){$x$}
		\tkzLabelAngle[pos=.9](I1,I,A){$40\dg$}
		\tkzLabelPoints[below left](A)
		\tkzLabelPoints[above right](B)
		\tkzLabelPoints[below right](C)
	\end{tikzpicture}
\end{figure}
%%a03
\begin{task}
	* $60\dg$
	* $80\dg$
	* $75\dg$
	* $120\dg$
\end{task}
%%r03
$80\dg$
%%p04
Calcular el valor de ``$x$'', en:
\begin{figure}[h]
	\begin{tikzpicture}[thick]
		\def\r{3.5}
		\def\a{9}
		\tkzDefPoint(-170:\r){A}
		\tkzDefPoint(10*\a-10:\r){B}
		\tkzDefPoint(-10:\r){C}
		\tkzDefShiftPoint[A](3*\a:1){I1}
		\tkzDefShiftPoint[C](120+3*\a:1){I2}
		\tkzInterLL(A,I1)(C,I2) \tkzGetPoint{I}
		\tkzFillAngles[size=6mm,fill=green,opacity=.2](A,B,C A,I,C B,C,I I,C,A C,A,I I,A,B)
		\tkzMarkAngles[size=6mm,mark=none](A,B,C A,I,C B,C,I I,C,A C,A,I I,A,B)
		\tkzDrawPolygon(A,B,C)
		\tkzDrawPolySeg(A,I,C)
		\tkzLabelAngle[pos=1](A,B,C){$80\dg$}
		\tkzLabelAngle[pos=.9](A,I,C){$x$}
		\tkzLabelAngle[pos=1.3,rotate=3*\a-60](B,C,I){$2\theta$}
		\tkzLabelAngle[pos=1.2](I,C,A){$3\theta$}
		\tkzLabelAngle[pos=1.2](C,A,I){$3\alpha$}
		\tkzLabelAngle[pos=1.4,rotate=3*\a](I,A,B){$2\alpha$}
	\end{tikzpicture}
\end{figure}
%%a04
\begin{task}
	* $100\dg$
	* $120\dg$
	* $130\dg$
	* $140\dg$
\end{task}
%%r04
$120\dg$
%%p05
Si $ABCD$ es un paralelogramo, calcular ``$x$''.
\begin{figure}[h]
	\begin{tikzpicture}[thick]
		\def\x{4}
		\def\y{3.8}
		\tkzDefPoints{0/0/A,\x/0/D}
		\tkzDefPoint(144:\y){B}
		\tkzDefParallelogram(D,A,B) \tkzGetPoint{C}
		\tkzFillAngles[size=5mm,fill=green,opacity=.2](B,C,D C,D,A)
		\tkzMarkAngles[size=5mm,mark=none](B,C,D C,D,A)
		\tkzLabelAngle[pos=.8](B,C,D){$4x$}
		\tkzLabelAngle(C,D,A){$x$}
		\tkzDrawPolygon(A,B,C,D)
		\tkzLabelPoints[below left](A)
		\tkzLabelPoints[above left](B)
		\tkzLabelPoints[above right](C)
		\tkzLabelPoints[below right](D)
	\end{tikzpicture}
\end{figure}
%%a05
\begin{task}
	* $24\dg$
	* $36\dg$
	* $20\dg$
	* $72\dg$
\end{task}
%%r05
$36\dg$
%%p06
\begin{tabular}{c}
	Si $ABCDEF$ es un hex\'agono regular, calcular ``$x$''. \vspace{5pt} \\
	\begin{tikzpicture}[thick]
		\def\r{2}
		\tkzDefPoints{1.1*\r/0/X,-1.1*\r/0/Y,0/\r/O,0/0/P1}
		\tkzDefRegPolygon[sides=6](O,P1)
		\tkzFillAngles[size=5mm,fill=green,opacity=.2](X,P1,P2 P6,P1,Y)
		\tkzMarkAngles[size=5mm,mark=none](X,P1,P2 P6,P1,Y)
		\tkzDrawPolygon(P1,P...,P6)
		\tkzDrawSegment[<->,>=latex](X,Y)
		\tkzLabelAngles[pos=.9](X,P1,P2 P6,P1,Y){$x$}
		\tkzLabelPoint[below](P1){$A$}
		\tkzLabelPoint[below left](P6){$B$}
		\tkzLabelPoint[above left](P5){$C$}
		\tkzLabelPoint[above](P4){$D$}
		\tkzLabelPoint[above right](P3){$E$}
		\tkzLabelPoint[below right](P2){$F$}
	\end{tikzpicture}
\end{tabular}
%%a06
\begin{task}
	* $30\dg$
	* $15\dg$
	* $20\dg$
	* $45\dg$
\end{task}
%%r06
$30\dg$
%%p07
\begin{tabular}{c}
	Si $ABCDE\dots$ es un pol\'igono de $20$ lados, calcular ``$x$''. \vspace{5pt} \\
	\begin{tikzpicture}[thick]
		\def\r{2.5}
		\def\a{1.4}
		\tkzDefPoints{-\a*\r/0/X,\a*\r/0/Y,0/-2.1*\r/O,0/.35*\r/P1}
		\tkzDefRegPolygon[sides=20](O,P1)
		\tkzInterLL(X,Y)(P19,P20) \tkzGetPoint{T1}
		\tkzInterLL(X,Y)(P2,P3) \tkzGetPoint{T2}
		\tkzFillAngles[size=5mm,fill=green,opacity=.2](P20,T1,X Y,T2,P2)
		\tkzMarkAngles[size=5mm,mark=none](P20,T1,X Y,T2,P2)
		\tkzDrawPolySeg(P18,P19,P20,P1,P2,P3,P4)
		\tkzDrawSegment(X,Y)
		\tkzLabelAngles[pos=.9](P20,T1,X Y,T2,P2){$x$}
		\tkzLabelPoint[above left](P3){$A$}
		\tkzLabelPoint[above left](P2){$B$}
		\tkzLabelPoint[above](P1){$C$}
		\tkzLabelPoint[above right](P20){$D$}
		\tkzLabelPoint[above right](P19){$E$}
	\end{tikzpicture}
\end{tabular}
%%a07
\begin{task}
	* $38\dg$
	* $18\dg$
	* $36\dg$
	* $27\dg$
\end{task}
%%r07
$27\dg$
%%p08.sp
\begin{mini}
	¿Cu\'al es el n\'umero de lados de aquel pol\'igono regular cuyo \'angulo interior es $2$ veces su \'angulo exterior?
\end{mini}
%%a08
\begin{enum}
	* $4$
	* $12$
	* $6$
	* $10$
\end{enum}
%%r08
$6$
%%p09.sp
\begin{mini}
	Si se triplica el n\'umero de lados de un pol\'igono, la suma de sus \'angulos internos queda quintuplicada. ¿C\'omo se llama dicho pol\'igono?
\end{mini}
%%a09
\begin{enum}
	* Tri\'angulo
	* Ala\'in
	* Cuadril\'atero
	* Oct\'ogono
\end{enum}
%%r09
Cuadril\'atero
%%p10.sp
\begin{mini}
	En un pol\'igono regular, el doble del n\'umero de diagonales es $5$ veces del n\'umero de lados. Luego, la medida de su \'angulo interior es:
\end{mini}
%%a10
\begin{task}
	* $120\dg$
	* $135\dg$
	* $180\dg$
	* $105\dg$
\end{task}
%%r10
$135\dg$
%%p11
Calcula el valor de ``$x$''.
\begin{figure}[h]
	\begin{tikzpicture}[thick]
		\def\r{3}
		\tkzDefPoints{0/0/O,1.1*\r/0/A}
		\tkzDefPoint(106:\r){B}
		\tkzDefParallelogram(B,O,A) \tkzGetPoint{C}
		\tkzFillAngles[size=4mm,fill=green,opacity=.2](A,O,B O,B,C)
		\tkzMarkAngles[size=4mm,mark=none](A,O,B O,B,C)
		\tkzLabelAngle[shift={(.5,-.4)}](A,O,B){$4x+54\dg$}
		\tkzLabelAngle[shift={(.3,.2)}](O,B,C){$8x-30\dg$}
		\tkzDrawPolygon(B,O,A,C)
	\end{tikzpicture}
\end{figure}
%%a11
\begin{task}
	* $12\dg$
	* $13\dg$
	* $16\dg$
	* $28\dg$
\end{task}
%%r11
$13\dg$
%%p12
\begin{tabular}{c}
	La medida del \'angulo exterior marcado en este dibujo es... \vspace{5pt} \\
	\begin{tikzpicture}[thick]
		\def\r{3}
		\tkzDefPoint(-\r,0){A}
		\tkzDefPoint(118:\r){B}
		\tkzDefPoint(-100:\r){C}
		\tkzFillAngles[size=5mm,fill=green,opacity=.2](B,A,C A,B,C B,C,A)
		\tkzMarkAngles[size=5mm,mark=none](B,A,C A,B,C B,C,A)
		\tkzDrawPolygon(A,B,C)
		\tkzLabelAngle(A,B,C){$40\dg$}
		\tkzLabelAngle[pos=1.3](B,C,A){$31\dg$}
	\end{tikzpicture}
\end{tabular}
%%a12
\begin{task}
	* $71\dg$
	* $261\dg$
	* $251\dg$
	* $109\dg$
\end{task}
%%r12
$251\dg$
%%p13.sp
\begin{mini}
	Si la relaci\'on entre el \'angulo interior y central de un pol\'igono regular es como $3$ a $2$, hallar el n\'umero de lados del pol\'igono.
\end{mini}
%%a13
\begin{task}
	* $3$
	* $2$
	* $5$
	* $4$
\end{task}
%%r13
$5$
%%p14
Calcula el valor de ``$x$'' del gr\'afico.
%%a14
\begin{task}
	* $30\dg$
	* $35\dg$
	* $70\dg$
	* $110\dg$
\end{task}
%%r14
$35\dg$
%%p15
Si el pol\'igono $ABCDE$ es regular, calcula el valor de ``$x$''.
%%a15
\begin{task}
	* $108\dg$
	* $36\dg$
	* $54\dg$
	* $72\dg$
\end{task}
%%r15
$54\dg$
