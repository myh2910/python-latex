%%p01.sp
\begin{mini}
	Si $CD=\ce{10u}$, hallar la longitud del segmento que une los puntos medios de $\ol{AC}$ y $\ol{BD}$.
\end{mini}
%%a01
\begin{task}
	* $\ce{4u}$
	* $\ce{3u}$
	* $\ce{3,5u}$
	* $\ce{6u}$
\end{task}
%%r01
$\ce{4u}$
%%p02
En la figura, $AB=BC=CD$. Calcular el valor de ``$\alpha$''.
%%a02
\begin{task}
	* $32\dg$
	* $68\dg$
	* $44\dg$
	* $70\dg$
\end{task}
%%r02
$68\dg$
%%p03
En la figura, calcular ``$x$''.
%%a03
\begin{task}
	* $60\dg$
	* $80\dg$
	* $75\dg$
	* $120\dg$
\end{task}
%%r03
$80\dg$
%%p04
Calcular el valor de ``$x$'', en:
%%a04
\begin{task}
	* $100\dg$
	* $120\dg$
	* $130\dg$
	* $140\dg$
\end{task}
%%r04
$130\dg$
%%p05
Si $ABCD$ es un paralelogramo, calcular ``$x$''.
%%a05
\begin{task}
	* $24\dg$
	* $36\dg$
	* $20\dg$
	* $72\dg$
\end{task}
%%r05
$36\dg$
%%p06
En la figura, $\ol{BC}\parallel\ol{AD}$. Calcular ``$\alpha$'' y ``$\beta$''.
\begin{figure}[h]
	\begin{tikzpicture}[thick]
		\def\x{3}
		\def\y{.8}
		\tkzDefPoints{\y/0/O1,-\y/0/O2}
		\tkzDefPoint(110:\x){B1}
		\tkzDefPoint(70:\x){C1}
		\tkzDefParallelogram(C1,O,O1) \tkzGetPoint{C}
		\tkzDefParallelogram(O2,O,B1) \tkzGetPoint{B}
		\tkzDefParallelogram(O1,C,B) \tkzGetPoint{A}
		\tkzDefParallelogram(C,B,O2) \tkzGetPoint{D}
		\tkzFillAngles[size=4mm,fill=green,opacity=.2](D,A,B A,B,C B,C,D C,D,A)
		\tkzMarkAngles[size=4mm,mark=none](D,A,B A,B,C B,C,D C,D,A)
		\tkzLabelAngle[pos=1](D,A,B){$70\dg$}
		\tkzLabelAngle[pos=1](A,B,C){$\beta$}
		\tkzLabelAngle[pos=1](B,C,D){$2\alpha-5\dg$}
		\tkzLabelAngle[pos=1](C,D,A){$\alpha+5\dg$}
		\tkzDrawPolygon(A,B,C,D)
		\tkzLabelPoints[below left](A)
		\tkzLabelPoints[above left](B)
		\tkzLabelPoints[above right](C)
		\tkzLabelPoints[below right](D)
	\end{tikzpicture}
\end{figure}
%%a06
\begin{mini}[1.4]
	\begin{task}
		* $124\dg$ y $12\dg$
		* $36\dg$ y $20\dg$
		* $120\dg$ y $70\dg$
		* $110\dg$ y $60\dg$
	\end{task}
\end{mini}
%%r06
$110\dg$ y $60\dg$
%%p08.sp
\begin{mini}
	¿Cu\'al es el n\'umero de lados de aquel pol\'igono regular cuyo \'angulo interior es $2$ veces su \'angulo exterior?
\end{mini}
%%r08
$6$
%%p09.sp
\begin{mini}
	Si se triplica el n\'umero de lados de un pol\'igono, la suma de sus \'angulos internos queda quintuplicada. ¿C\'omo se llama dicho pol\'igono?
\end{mini}
%%r09
Cuadrado
%%p10.sp
\begin{mini}
	En un pol\'igono regular, el doble del n\'umero de diagonales es $5$ veces del n\'umero de lados. Luego, la medida de su \'angulo interior es:
\end{mini}
%%a10
\begin{task}
	* $120\dg$
	* $135\dg$
	* $180\dg$
	* $105\dg$
\end{task}
%%r10
$135\dg$
%%p11
Calcula el valor de ``$x$''.
\begin{figure}[h]
	\begin{tikzpicture}[thick]
		\def\r{3}
		\tkzDefPoints{0/0/O,1.1*\r/0/A}
		\tkzDefPoint(106:\r){B}
		\tkzDefParallelogram(B,O,A) \tkzGetPoint{C}
		\tkzFillAngles[size=4mm,fill=green,opacity=.2](A,O,B O,B,C)
		\tkzMarkAngles[size=4mm,mark=none](A,O,B O,B,C)
		\tkzLabelAngle[shift={(.5,-.4)}](A,O,B){$4x+54\dg$}
		\tkzLabelAngle[shift={(.3,.2)}](O,B,C){$8x-30\dg$}
		\tkzDrawPolygon(B,O,A,C)
	\end{tikzpicture}
\end{figure}
%%r11
$13\dg$
%%p13.sp
\begin{mini}
	Si la relaci\'on entre el \'angulo interior y central de un pol\'igono regular es como $3$ a $2$, hallar el n\'umero de lados del pol\'igono.
\end{mini}
%%a13
\begin{task}
	* $3$
	* $2$
	* $5$
	* $4$
\end{task}
%%r13
$5$
%%p14
Calcula el valor de ``$x$'' del gr\'afico.
%%a14
\begin{task}
	* $30\dg$
	* $35\dg$
	* $70\dg$
	* $110\dg$
\end{task}
%%r14
$35\dg$
%%p15
Si el pol\'igono $ABCDE$ es regular, calcula el valor de ``$x$''.
%%a15
\begin{task}
	* $108\dg$
	* $36\dg$
	* $54\dg$
	* $72\dg$
\end{task}
%%r15
$54\dg$
