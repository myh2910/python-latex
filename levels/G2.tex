%%p01.sp
\begin{mini}
	Escribe verdadero (V) o falso (F) seg\'un corresponda y encierra la alternativa correcta.
	\begin{center}\vspace{-12pt}
		\begin{tabular}{m{.01\linewidth}m{.77\linewidth}l}
			I.&Un hex\'agono posee $9$ diagonales.&(\phantom{V}) \vspace{4pt}\\
			II.&Un pent\'agono posee $5$ diagonales.&(\phantom{V}) \vspace{4pt}\\
			III.&La suma de las medidas de \'angulos&\multirow{2}{*}{(\phantom{V})} \\
			&interiores de un dodec\'agono es $1800\dg$.&
		\end{tabular}
	\end{center}
\end{mini}
%%a01
\begin{task}
	* VVF
	* FFV
	* VVV
	* FFF
\end{task}
%%r01
VVV
%%p02
\begin{tabular}{c}
	La medida del \'angulo exterior marcado en este dibujo es... \vspace{5pt} \\
	\begin{tikzpicture}[thick]
		\def\r{3}
		\tkzDefPoint(-\r,0){A}
		\tkzDefPoint(118:\r){B}
		\tkzDefPoint(-100:\r){C}
		\tkzFillAngles[size=5mm,fill=yellow,opacity=.2](B,A,C A,B,C B,C,A)
		\tkzMarkAngles[size=5mm,mark=none](B,A,C A,B,C B,C,A)
		\tkzDrawPolygon(A,B,C)
		\tkzLabelAngle(A,B,C){$40\dg$}
		\tkzLabelAngle[pos=1.3](B,C,A){$31\dg$}
	\end{tikzpicture}
\end{tabular}
%%a02
\begin{task}
	* $71\dg$
	* $261\dg$
	* $251\dg$
	* $109\dg$
\end{task}
%%r02
$251\dg$
%%p03
\begin{mini}
	Hallar la medida del \'angulo interior del pol\'igono regular mostrado.
	\begin{center}
		\begin{tikzpicture}[thick]
			\tkzDefPoints{0/0/P0,0/2/P1}
			\tkzDefRegPolygon[sides=5](P0,P1)
			\tkzDrawPolygon(P1,P...,P5)
		\end{tikzpicture}
	\end{center}
\end{mini}
%%a03
\begin{task}
	* $92\dg$
	* $120\dg$
	* $108\dg$
	* $140\dg$
\end{task}
%%r03
$108\dg$
%%p04.sp
\begin{tabular}{c}
	Si la figura es un pol\'igono regular, hallar ``$x$''. \vspace{5pt} \\
	\begin{tikzpicture}[thick]
		\tkzDefPoint(0,0){P0}
		\tkzDefPoint(112.5:2){P1}
		\tkzDefRegPolygon[sides=8](P0,P1)
		\tkzFillAngle[size=3mm,fill=yellow,opacity=.2](P3,P2,P1)
		\tkzMarkAngle[size=3mm,mark=none](P3,P2,P1)
		\tkzLabelAngle[pos=.6](P3,P2,P1){$x$}
		\tkzDrawPolygon(P1,P...,P8)
	\end{tikzpicture}
\end{tabular}
%%a04
\begin{task}
	* $120\dg$
	* $145\dg$
	* $140\dg$
	* $135\dg$
\end{task}
%%r04
$135\dg$
%%p05
\begin{tabular}{c}
	Si $ABCDE$ es un pol\'igono regular, calcula el valor de ``$x$''. \vspace{5pt} \\
	\begin{tikzpicture}[thick]
		\tkzDefPoints{0/0/O,0/2.5/P1}
		\tkzDefRegPolygon[sides=5](O,P1)
		\tkzFillAngles[size=5mm,fill=yellow,opacity=.2](P1,P5,P4)
		\tkzMarkAngles[size=5mm,mark=none](P1,P5,P4)
		\tkzDrawPolygon(P1,P...,P5)
		\tkzLabelAngle(P1,P5,P4){$2x$}
		\tkzLabelPoint[below left](P3){$A$}
		\tkzLabelPoint[above left](P2){$B$}
		\tkzLabelPoint[above](P1){$C$}
		\tkzLabelPoint[above right](P5){$D$}
		\tkzLabelPoint[below right](P4){$E$}
	\end{tikzpicture}
\end{tabular}
%%a05
\begin{task}
	* $108\dg$
	* $36\dg$
	* $54\dg$
	* $72\dg$
\end{task}
%%r05
$54\dg$
%%p06
Calcular ``$x$''.
\begin{figure}[h]
	\begin{tikzpicture}[thick]
		\tkzDefPoints{0/0/A,0/4/B,-6.4/0/C}
		\tkzDefPointBy[rotation=center B angle -81](A) \tkzGetPoint{E}
		\tkzDefPointBy[rotation=center C angle 54](A) \tkzGetPoint{F}
		\tkzInterLL(B,E)(C,F) \tkzGetPoint{D}
		\tkzMarkRightAngle[size=.3,fill=yellow!20](B,A,C)
		\def\r{5mm}
		\tkzFillAngles[size=\r,fill=yellow,opacity=.2](D,B,A A,C,D C,D,B)
		\tkzMarkAngles[size=\r,mark=none](D,B,A A,C,D C,D,B)
		\tkzLabelAngle(D,B,A){$3x$}
		\tkzLabelAngle(A,C,D){$2x$}
		\tkzLabelAngle(C,D,B){$5x$}
		\tkzDrawPolygon(B,A,C,D)
	\end{tikzpicture}
\end{figure}
%%a06
\begin{task}
	* $13\dg$
	* $17\dg$
	* $14\dg$
	* $27\dg$
\end{task}
%%r06
$27\dg$
%%p07
En la figura, calcular $\alpha$.
\begin{figure}[h]
	\begin{tikzpicture}[thick]
		\def\r{2.5}
		\tkzDefPoint(-160:\r){A}
		\tkzDefPoint(120:\r){B}
		\tkzDefPoint(-30:\r){C}
		\def\a{6mm}
		\tkzFillAngles[size=\a,fill=yellow,opacity=.2](C,A,B A,B,C B,C,A)
		\tkzMarkAngles[size=\a,mark=none](C,A,B A,B,C B,C,A)
		\tkzLabelAngle(C,A,B){$75\dg$}
		\tkzLabelAngle(A,B,C){$\alpha$}
		\tkzLabelAngle(B,C,A){$40\dg$}
		\tkzDrawPolygon(A,B,C)
		\tkzLabelPoints[below left](A)
		\tkzLabelPoints[above](B)
		\tkzLabelPoints[below right](C)
	\end{tikzpicture}
\end{figure}
%%a07
\begin{task}
	* $80\dg$
	* $40\dg$
	* $65\dg$
	* $150\dg$
\end{task}
%%r07
$65\dg$
%%p08
Calcula el valor de ``$x$''.
\begin{figure}[h]
	\begin{tikzpicture}[thick]
		\def\r{2.5}
		\tkzDefPoint(-22:\r){v1}
		\tkzDefPoint(-158:\r){v2}
		\tkzDefPoint(66:\r){v3}
		\tkzFillAngles[size=6mm,fill=yellow,opacity=.2](v1,v2,v3 v2,v3,v1)
		\tkzMarkAngles[size=6mm,mark=none](v1,v2,v3 v2,v3,v1)
		\tkzLabelAngle(v1,v2,v3){$x$}
		\tkzLabelAngle(v2,v3,v1){$68\dg$}
		\tkzDrawPolygon(v1,v2,v3)
		\tkzMarkSegments[mark=||](v1,v2 v2,v3)
	\end{tikzpicture}
\end{figure}
%%a08
\begin{task}
	* $34\dg$
	* $41\dg$
	* $44\dg$
	* $49\dg$
\end{task}
%%r08
$44\dg$
%%p09
Calcular ``$\theta$'' en:
\begin{figure}[h]
	\begin{tikzpicture}[thick]
		\def\r{4.5}
		\def\a{5mm}
		\tkzDefPoint(-160:\r){A}
		\tkzDefPoint(164:\r){B}
		\tkzDefPoint(-12:\r){C}
		\tkzFillAngles[size=\a,fill=yellow,opacity=.2](C,A,B A,B,C)
		\tkzFillAngle[size=8mm,fill=yellow,opacity=.2](B,C,A)
		\tkzMarkAngles[size=\a,mark=none](C,A,B A,B,C)
		\tkzMarkAngle[size=8mm,mark=none](B,C,A)
		\tkzLabelAngle[shift={(0,-.1)}](C,A,B){$2\theta$}
		\tkzLabelAngle[shift={(.2,0)}](A,B,C){$\theta+30\dg$}
		\tkzLabelAngle[pos=1.7](B,C,A){$18\dg$}
		\tkzDrawPolygon(A,B,C)
	\end{tikzpicture}
\end{figure}
%%a09
\begin{task}
	* $42\dg$
	* $45\dg$
	* $44\dg$
	* $48\dg$
\end{task}
%%r09
$44\dg$
%%p10
\begin{tabular}{c}
	Si el tri\'angulo $ABC$ es equil\'atero, calcular $\text{m}\dang BDC$. \vspace{5pt} \\
	\begin{tikzpicture}[thick]
		\def\r{5}
		\tkzDefPoint(-120:\r){A}
		\tkzDefPoint(0,0){B}
		\tkzDefPoint(-60:\r){C}
		\tkzDefPoint(-100:1){P}
		\tkzInterLL(B,P)(A,C)
		\tkzGetPoint{D}
		\tkzFillAngle[size=8mm,fill=yellow,opacity=.2](A,B,D)
		\tkzMarkAngle[size=8mm,mark=none](A,B,D)
		\tkzLabelAngle[pos=1.7](A,B,D){$20\dg$}
		\tkzDrawPolygon(A,B,C)
		\tkzDrawSegment(B,D)
		\tkzLabelPoints[below left](A)
		\tkzLabelPoints[above](B)
		\tkzLabelPoints[below right](C)
		\tkzLabelPoints[below](D)
	\end{tikzpicture}
\end{tabular}
%%a10
\begin{task}
	* $70\dg$
	* $80\dg$
	* $90\dg$
	* $75\dg$
\end{task}
%%r10
$80\dg$
%%p11
Calcula el valor de ``$x$'' en el gr\'afico mostrado.
\begin{figure}[h]
	\begin{tikzpicture}[thick]
		\def\r{3}
		\tkzDefPoints{-\r/\r/A,-\r/0/B,0/-\r/C1,2*\r/-\r/D1,0/\r/E}
		\def\k{1.2}
		\tkzDefBarycentricPoint(B=1,C1=\k) \tkzGetPoint{C}
		\tkzDefBarycentricPoint(E=1,D1=1+2*\k) \tkzGetPoint{D}
		\tkzMarkRightAngle[size=.4,fill=yellow!20](B,A,E)
		\def\a{5mm}
		\tkzFillAngles[size=\a,fill=yellow,opacity=.2](C,B,A D,C,B E,D,C A,E,D)
		\tkzMarkAngles[size=\a,mark=none](C,B,A D,C,B E,D,C A,E,D)
		\tkzDrawPolygon(A,B,C,D,E)
		\tkzLabelAngles(C,B,A D,C,B A,E,D){$3x$}
		\tkzLabelAngle(E,D,C){$x$}
	\end{tikzpicture}
\end{figure}
%%a11
\begin{task}
	* $30\dg$
	* $45\dg$
	* $54\dg$
	* $36\dg$
\end{task}
%%r11
$45\dg$
%%p12.sp
\begin{mini}
	En un tri\'angulo is\'osceles, la suma de dos \'angulos distintos es igual a $110\dg$. Entonces la suma de los \'angulos de la base es:
\end{mini}
%%a12
\begin{task}
	* $150\dg$
	* $146\dg$
	* $140\dg$
	* $136\dg$
\end{task}
%%r12
$140\dg$
%%p13.sp
\begin{mini}
	Si $CD=\ce{10u}$, hallar la longitud del segmento que une los puntos medios de $\ol{AC}$ y $\ol{BD}$.
	\begin{center}
		\begin{tikzpicture}[thick]
			\def\r{2.5}
			\tkzDefPoints{0/0/A,0/\r/B,\r/\r/C,1/0/D1}
			\tkzDefPointBy[rotation=center C angle 143](B) \tkzGetPoint{D2}
			\tkzInterLL(A,D1)(C,D2) \tkzGetPoint{D}
			\tkzMarkRightAngles[size=.3,fill=yellow!20](D,A,B A,B,C)
			\tkzFillAngle[size=6mm,fill=yellow,opacity=.2](C,D,A)
			\tkzMarkAngle[size=6mm,mark=none](C,D,A)
			\tkzLabelAngle[pos=1.1](C,D,A){$37\dg$}
			\tkzDrawPolygon(A,B,C,D)
			\tkzLabelPoints[below left](A)
			\tkzLabelPoints[above left](B)
			\tkzLabelPoints[above right](C)
			\tkzLabelPoints[below right](D)
		\end{tikzpicture}
	\end{center}
\end{mini}
%%a13
\begin{task}
	* $\ce{4u}$
	* $\ce{3u}$
	* $\ce{3,5u}$
	* $\ce{6u}$
\end{task}
%%r13
$\ce{4u}$
%%p14.sp
\begin{mini}[.75]
	En un $\triangle ABC$, $AB=BC$ y $\text{m}\dang B=108\dg$. Calcular la medida del \'angulo exterior en el v\'ertice ``$C$''.
	\begin{center}
		\begin{tikzpicture}[thick]
			\def\r{3.6}
			\tkzDefPoint(162:\r){A}
			\tkzDefPoint(0,\r){B}
			\tkzDefPoint(18:\r){C}
			\tkzDefPointOnLine[pos=1.2](A,C)
			\tkzGetPoint{D}
			\tkzFillAngles[size=5mm,fill=yellow,opacity=.2](A,B,C B,C,A C,A,B)
			\tkzFillAngle[size=3mm,fill=yellow,opacity=.2](D,C,B)
			\tkzMarkAngles[size=5mm,mark=none](A,B,C B,C,A C,A,B)
			\tkzMarkAngle[size=3mm,mark=none](D,C,B)
			\tkzLabelAngles[pos=.8](B,C,A C,A,B){$\alpha$}
			\tkzLabelAngle[pos=.8](A,B,C){$108\dg$}
			\tkzLabelAngle[pos=.55](D,C,B){$x$}
			\tkzDrawPolySeg(D,A,B,C)
			\tkzDefMidPoint(A,B)
			\tkzGetPoint{M}
			\tkzDefMidPoint(B,C)
			\tkzGetPoint{N}
			\tkzDrawPoints[size=6.5,color=black,fill=yellow!20](M,N)
			\tkzLabelPoints[below left](A)
			\tkzLabelPoints[above](B)
			\tkzLabelPoints[below](C)
		\end{tikzpicture}
	\end{center}
\end{mini}
%%a14
\begin{task}
	* $89\dg$
	* $124\dg$
	* $136\dg$
	* $144\dg$
\end{task}
%%r14
$144\dg$
%%p15.sp
\begin{mini}
	Calcula el n\'umero total de diagonales que se pueden trazar en la siguiente figura:
	\begin{center}
		\begin{tikzpicture}[thick]
			\def\x{1.4}
			\def\y{1.5}
			\tkzDefPoints{-\x/0/A,\x/0/B}
			\tkzDefSquare(A,B)
			\tkzGetPoints{C}{D}
			\tkzDefPoints{-\x/\y/E,\x/\y/F}
			\tkzDefEquilateral(A,E)
			\tkzGetPoint{G}
			\tkzDefEquilateral(F,B)
			\tkzGetPoint{H}
			\tkzDrawPolygon(A,B,F,H,C,D,G,E)
		\end{tikzpicture}
	\end{center}
\end{mini}
%%a15
\begin{task}
	* $21$
	* $23$
	* $28$
	* $20$
\end{task}
%%r15
$20$
