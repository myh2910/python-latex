%%p01.sp
\begin{mini}
	Susana es una doctora que trabaja en un hospital de Huancayo. Ella utiliza $2\dfrac{1}{2}$ litros de suero para un paciente y $1\dfrac{2}{6}$ litros para otro. Calcula cu\'antos litros de suero utiliza.
\end{mini}
%%a01
\begin{task}
	* $\dfrac{23}{7}$
	* $\dfrac{23}{6}$
	* $-\dfrac{23}{6}$
	* $\dfrac{6}{23}$
\end{task}
%%r01
$\dfrac{23}{6}$
%%p02.sp
\begin{mini}[.8]
	Tres amigas: Carmen, F\'atima y Milagros, comentan sobre el color de polo que llevan puesto.
	\begin{itemize}
		\ii Carmen dice: ``Mi polo no es rojo ni azul como los de ustedes''.
		\ii Milagros dice: ``Me gustar\'ia tener un polo verde como el tuyo''.
		\ii F\'atima dice: ``Me gusta mi polo rojo''.
	\end{itemize}
	¿Qu\'e color de polo tiene Carmen?
\end{mini}
%%a02
\begin{mini}[.7]
	\begin{enum*}
		* Rojo
		* Azul
		* Verde
		* Rosado
	\end{enum*}
\end{mini}
%%r02
Verde
%%p03.sp
\begin{mini}[.7]
	Seis amigos viven en un edificio, cada uno en un piso diferente. Carlos vive m\'as abajo que Bica, pero m\'as arriba que David. Franco vive $3$ pisos m\'as abajo que Carlos. Andr\'es vive $2$ pisos m\'as arriba que Carlos y a $4$ pisos de Enzo. ¿El tercer piso lo ocupa?
\end{mini}
%%a03
\begin{task}
	* Bica
	* David
	* Franco
	* Carlos
\end{task}
%%r03
David
%%p04.sp
\begin{mini}[.8]
	Seis amigos se sientan alrededor de una mesa circular en seis asientos sim\'etricamente distribuidos. Se conoce lo siguiente:
	\begin{itemize}
		\ii Ernesto est\'a al frente de Carla.
		\ii Dina est\'a al frente de Flor, quien no est\'a junto a Alonso.
		\ii Carla est\'a junto y a la derecha de Alonso.
	\end{itemize}
	¿Qu\'ien est\'a junto y a la izquierda de Beto?
\end{mini}
%%a04
\begin{mini}[.7]
	\begin{enum*}
		* Carla
		* Flor
		* Dina
		* Ernesto
	\end{enum*}
\end{mini}
%%r04
Flor
%%p05.sp
\begin{mini}[.8]
	Milton, N\'estor, Pa\'ul y Renzo se sentaron a beber alrededor de una mesa circular, en asientos sim\'etricamente dispuestos. El que se sent\'o a la izquierda de N\'estor bebi\'o agua; Milton estaba frente al que beb\'ia vino. Quien se sentaba a la derecha de Renzo, beb\'ia gaseosa, el que beb\'ia caf\'e y el que beb\'ia gaseosa estaban frente a frente. ¿Qu\'e beb\'ia Pa\'ul y qui\'en tomaba vino respectivamente?
\end{mini}
%%a05
\begin{enum}
	* Gaseosa - Milton
	* Vino - Pa\'ul
	* Gaseosa - Renzo
	* Agua - Milton
\end{enum}
%%r05
Gaseosa - Renzo
%%p06.sp
\begin{mini}[.9]
	En una reuni\'on se encuentran $6$ amigos: Carlos, N\'estor, Juana, Lizandro, Mar\'ia y Victoria, quienes se sientan alternadamente varones y mujeres en seis sillas igualmente espaciadas alrededor de una mesa circular. Se sabe lo siguiente:
	\begin{itemize}
		\ii N\'estor se sienta a la derecha y junto a Victoria.
		\ii Carlos se sienta frente a Victoria.
	\end{itemize}
	¿Cu\'ales de las siguientes afirmaciones son correctas?
	\begin{enumerate}[label=\Roman*)]
		\ii N\'estor se sienta junto a Mar\'ia.
		\ii Lizandro se sienta junto a Victoria.
		\ii Mar\'ia no se sienta frente a Carlos.
	\end{enumerate}
\end{mini}
%%a06
\begin{task}
	* I y II
	* II y III
	* I y III
	* Todas
\end{task}
%%r06
II y III
%%p07.sp
\begin{mini}[.7]
	Carlos, Dante, Toño, Erick, Beto y Flavio se ubican en $6$ asientos contiguos en una hilera de un teatro. Toño est\'a junto y a la izquierda de Beto, Carlos a la derecha de Toño entre Flavio y Dante, Dante est\'a junto y a la izquierda de Erick. ¿Qui\'en ocupa el tercer asiento si los contamos de izquierda a derecha?
\end{mini}
%%a07
\begin{mini}[.7]
	\begin{enum*}
		* Carlos
		* Erick
		* Dante
		* Flavio
	\end{enum*}
\end{mini}
%%r07
Flavio
%%p08.sp
\begin{mini}[.8]
	$3$ amigos estudiaron en la universidad; uno F\'isica, otro Agronom\'ia y otro Ingenier\'ia. Cada uno de ellos tiene un hijo que cuando ingrese a la universidad, decidir\'a no tomar la carrera de su padre sino estudiar la carrera de uno de los amigos de su padre. Sabiendo que el Ingeniero se llama Luis y que el hijo de Juan quiere ser Agr\'onomo; ¿qu\'e profesi\'on tiene Juan y a qu\'e quiere dedicarse el hijo de Rogelio?
\end{mini}
%%a08
\begin{enum}
	* Juan es Agr\'onomo y el hijo de Rogelio quiere ser F\'isico
	* Juan es F\'isico y el hijo de Rogelio quiere ser Ingeniero
	* Juan es Agr\'onomo y el hijo de Rogelio quiere ser Agr\'onomo
	* Juan es F\'isico y el hijo de Rogelio quiere ser F\'isico
\end{enum}
%%r08
\begin{tabular}{c}
	Juan es F\'isico y \\
	el hijo de Rogelio \\
	quiere ser Ingeniero
\end{tabular}
%%p09.sp
\begin{mini}[.8]
	Alrededor de una mesa circular est\'an sentadas $6$ amigas distribuidas sim\'etricamente. Si se sabe que:
	\begin{itemize}
		\ii Karen se ubica junto a Rosa pero no junto a Mar\'ia.
		\ii Ana se sienta frente a la persona que est\'a junto y a la izquierda de Rosa.
		\ii Mar\'ia est\'a a dos lugares de Ana.
		\ii In\'es se ubica a dos lugares a la derecha de Dora.
	\end{itemize}
	¿Qui\'en se encuentra frente a Ana?
\end{mini}
%%a09
\begin{task}
	* Dora
	* Rosa
	* Mar\'ia
	* In\'es
\end{task}
%%r09
In\'es
%%p10.sp
\begin{mini}[.8]
	En la playa cuatro niñas: Anita, Luisita, Carmencita y Dianita forman una circunferencia al tomarse de las manos. Si:
	\begin{itemize}
		\ii La niña de ropa de baño verde est\'a a la izquierda de Carmencita.
		\ii Luisita est\'a al frente de la niña de ropa de baño rojo.
		\ii La niña a la derecha de Anita tiene ropa de baño de color rosado, y \'esta se encuentra frente a la de ropa de baño amarilla.
	\end{itemize}
	¿De qu\'e color es la ropa de baño de Dianita?
\end{mini}
%%a10
\begin{mini}[.7]
	\begin{enum*}
		* Verde
		* Rojo
		* Rosado
		* Amarillo
	\end{enum*}
\end{mini}
%%r10
Rosado
%%p11.sp
\begin{mini}[.8]
	En un comedor de estudiantes, $8$ comensales se sientan en una mesa circular guardando distancias proporcionales. Las $8$ personas son estudiantes de diversas especialidades.
	\begin{itemize}
		\ii El de Ingenier\'ia est\'a frente al de Educaci\'on y entre (a los costados) los de Econom\'ia y Farmacia.
		\ii El de Periodismo est\'a a la izquierda de el de Educaci\'on y frente al de Econom\'ia.
		\ii Frente al de Farmacia est\'a el de Derecho; este a su vez est\'a a la siniestra de el de Arquitectura.
	\end{itemize}
	¿Cu\'al de ellos est\'a entre los estudiantes de Biolog\'ia y Educaci\'on?
\end{mini}
%%a11
\begin{mini}[.9]
	\begin{enum*}
		* Periodismo
		* Farmacia
		* Ingenier\'ia
		* Econom\'ia
	\end{enum*}
\end{mini}
%%r11
Periodismo
%%p12.sp
\begin{mini}[.8]
	Fernanda, Silvia y Ana llegan a una fiesta a las $\ce{8pm}$, $\ce{9pm}$ y $\ce{10pm}$ aunque no necesariamente en ese orden y llevan puestos vestidos de color rojo, negro y turquesa no necesariamente en ese orden. Se tiene la siguiente informaci\'on:
	\begin{itemize}
		\ii La que lleg\'o a las $\ce{9pm}$, lleva vestido de color rojo.
		\ii Silvia que lleg\'o \'ultima le comenta a la que lleva vestido negro, que ese color le queda espectacular.
	\end{itemize}
	Si Fernanda lleg\'o antes que todas, ¿qui\'en lleva el vestido negro y a qu\'e hora lleg\'o?
\end{mini}
%%a12
\begin{mini}
	\begin{enum}(2)
		* Ana - $\ce{8pm}$
		* Fernanda - $\ce{10pm}$
		* Silvia - $\ce{10pm}$
		* Fernanda - $\ce{8pm}$
	\end{enum}
\end{mini}
%%r12
Fernanda - $\ce{8pm}$
%%p13.sp
\begin{mini}[.8]
	Luz, Mar\'ia, Nancy, Ofelia, Pamela y Rita se sientan alrededor de una mesa circular con seis asientos distribuidos sim\'etricamente. Si se sabe que:
	\begin{itemize}
		\ii Ofelia no se sienta junto a Mar\'ia.
		\ii Pamela no se sienta junto a Nancy.
		\ii Luz se sienta junto y a la derecha de Mar\'ia y frente a Nancy.
	\end{itemize}
	¿D\'onde se sienta Rita?
\end{mini}
%%a13
\begin{mini}[.9]
	\begin{enum}(2)
		* Junto y entre Nancy y Pamela
		* Frente a Nancy
		* Junto y entre Mar\'ia y Nancy
		* Frente a Mar\'ia
	\end{enum}
\end{mini}
%%r13
\begin{tabular}{c}
	Junto y entre \\
	Mar\'ia y Nancy
\end{tabular}
%%p14.sp
\begin{mini}[.8]
	En un club se encuentran $4$ deportistas cuyos nombres son: Juan, Mario, Luis y Jorge. Los deportes que practican son nataci\'on, b\'asquet, f\'utbol y tenis. Cada uno juega solo un deporte. El nadador que es primo de Juan, es cuñado de Mario y adem\'as es el m\'as joven del grupo. Luis que es el de m\'as edad, es vecino del basquetbolista quien a su vez es un mujeriego empedernido. Juan que es sumamente t\'imido con las mujeres es $7$ años menor que el tenista. ¿Qui\'en practica b\'asquet?
\end{mini}
%%a14
\begin{task}
	* Juan
	* Mario
	* Luis
	* Jorge
\end{task}
%%r14
Mario
%%p15.sp
\begin{mini}[.8]
	Almorzaban juntos $3$ pol\'iticos: el señor blanco, el señor rojo y el señor amarillo; uno llevaba corbata blanca, otro corbata roja y otro corbata amarilla pero no necesariamente en ese orden. <<Es curioso>> dijo, el señor de la corbata roja: <<Nuestros apellidos son los mismos que nuestras corbatas, pero ninguno lleva la que corresponde al suyo>>. <<Tiene usted raz\'on>> dijo el señor blanco. ¿De qu\'e color llevaba la corbata el señor amarillo?
\end{mini}
%%a15
\begin{mini}[.6]
	\begin{enum}(2)
		* Blanco
		* Rojo
		* Amarillo
		* Blanco y Rojo
	\end{enum}
\end{mini}
%%r15
Rojo
