%%p01.90s
$\dfrac{2}{3}-\dfrac{3}{4}\cdot\dfrac{2}{3}+\dfrac{5}{6}$
%%a01
\begin{task}
	* $1$
	* $-1$
	* $2$
	* $\dfrac{1}{2}$
\end{task}
%%r01
$1$
%%p02.120s
$\left(\dfrac{1}{2}+\dfrac{5}{6}\right)\div\dfrac{4}{3}-\dfrac{1}{5}$
%%a02
\begin{task}
	* $\dfrac{4}{6}$
	* $-\dfrac{4}{6}$
	* $\dfrac{4}{5}$
	* $-\dfrac{4}{5}$
\end{task}
%%r02
$\dfrac{4}{5}$
%%p03.100s
$\dfrac{\dfrac{5}{6}+\dfrac{1}{2}}{\dfrac{2}{3}-\dfrac{5}{9}}$
%%a03
\begin{task}
	* $\dfrac{1}{12}$
	* $12$
	* $-\dfrac{1}{12}$
	* $-12$
\end{task}
%%r03
$12$
%%p04.100s
$\dfrac{\left(\dfrac{2}{3}\div\dfrac{3}{2}\right)}{\left(\dfrac{5}{3}\cdot\dfrac{1}{2}\right)}$
%%a04
\begin{task}
	* $\dfrac{8}{15}$
	* $\dfrac{4}{12}$
	* $\dfrac{15}{2}$
	* $-\dfrac{2}{15}$
\end{task}
%%r04
$\dfrac{8}{15}$
%%p05.120s
$\dfrac{\dfrac{1}{3}+\dfrac{1}{4}}{3\dfrac{1}{2}-2}$
%%a05
\begin{task}
	* $\dfrac{14}{36}$
	* $\dfrac{36}{14}$
	* $\dfrac{28}{14}$
	* $\dfrac{24}{36}$
\end{task}
%%r05
$\dfrac{14}{36}$
%%p06.200s
$\left[\dfrac{6}{5}\div\dfrac{9}{10}-\left(2-\dfrac{7}{12}\right)\right]+\dfrac{7}{24}$
%%a06
\begin{task}
	* $\dfrac{5}{34}$
	* $-\dfrac{5}{23}$
	* $\dfrac{5}{24}$
	* $\dfrac{5}{32}$
\end{task}
%%r06
$\dfrac{5}{24}$
%%p07.sp.30s
\begin{mini}[.6]
	V\'ictor, Daniel, Beto son militares con $3$ rangos distintos; soldado, cabo y mayor aunque no necesariamente en ese orden. Si:
	\begin{itemize}
		\ii Beto es el soldado
		\ii Daniel no es el cabo
	\end{itemize}
	¿C\'omo se llama el mayor?
\end{mini}
%%a07
\begin{mini}[.7]
	\begin{enum*}
		* V\'ictor
		* Daniel
		* Beto
		* F.D.
	\end{enum*}
\end{mini}
%%r07
Daniel
%%p08.sp.40s
\begin{mini}
	Karina recorre los $\dfrac{2}{3}$ de un camino en bicicleta y $\dfrac{1}{9}$ a pie. ¿Qu\'e parte del camino recorri\'o en total?
\end{mini}
%%a08
\begin{task}
	* $\dfrac{5}{7}$
	* $\dfrac{7}{9}$
	* $\dfrac{2}{3}$
	* $\dfrac{3}{7}$
\end{task}
%%r08
$\dfrac{7}{9}$
%%p09.sp.40s
\begin{mini}
	La raz\'on geom\'etrica de dos n\'umeros vale $\dfrac{4}{7}$ y su raz\'on aritm\'etica es $45$. Determina el menor de los n\'umeros.
\end{mini}
%%a09
\begin{task}
	* $50$
	* $45$
	* $60$
	* $52$
\end{task}
%%r09
$60$
%%p10.sp.80s
\begin{mini}
	La suma de dos n\'umeros es $144$ y su raz\'on geom\'etrica vale $\dfrac{2}{7}$. ¿Cu\'ales son los dichos n\'umeros?
\end{mini}
%%a10
\begin{enum}
	* $100$ y $44$
	* $81$ y $63$
	* $32$ y $122$
	* $90$ y $36$
\end{enum}
%%r10
$32$ y $122$
%%p11.sp.80s
\begin{mini}[.7]
	$6$ amigas se sientan alrededor de una mesa circular. Mar\'ia, que est\'a sentada a la derecha de Paola, se encuentra frente a Noem\'i. Paola est\'a frente a la que est\'a junto y a la derecha de Sara, que est\'a frente a Raquel. ¿Qui\'en est\'a junto y a la derecha de Carmen?
\end{mini}
%%a11
\begin{mini}[.7]
	\begin{enum*}
		* Raquel
		* Paola
		* Mar\'ia
		* Noem\'i
	\end{enum*}
\end{mini}
%%r11
Noem\'i
%%p12.sp.90s
Si $\dfrac{a}{b}=\dfrac{7}{11}$ y adem\'as $a\cdot b=308$, calcular $b-a$.
%%a12
\begin{task}
	* $2$
	* $3$
	* $7$
	* $8$
\end{task}
%%r12
$8$
%%p13.sp.90s
\begin{mini}
	Tres n\'umeros son entre s\'i como $4$, $7$ y $11$; y la suma del menor con el mayor de dichos n\'umeros es $105$. Determinar el menor de estos n\'umeros.
\end{mini}
%%a13
\begin{task}
	* $49$
	* $14$
	* $24$
	* $28$
\end{task}
%%r13
$28$
%%p14.sp.90s
\begin{mini}
	Las edades de Juan y Rocio est\'an en relaci\'on de $5$ a $9$ y la suma de ellas es $84$. ¿Qu\'e edad tiene Juan?
\end{mini}
%%a14
\begin{task}
	* $20$
	* $30$
	* $40$
	* $45$
\end{task}
%%r14
$30$
%%p15.sp.90s
\begin{mini}
	Tres n\'umeros est\'an en la misma relaci\'on que $5$, $9$ y $13$. Si la suma de ellos es $216$, indica el mayor de ellos.
\end{mini}
%%a15
\begin{task}
	* $102$
	* $88$
	* $104$
	* $96$
\end{task}
%%r15
$104$
